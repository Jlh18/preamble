\documentclass{article}

\usepackage[left=1in,right=1in]{geometry}
\usepackage{subfiles}
\usepackage{amsmath, amssymb, stmaryrd, verbatim} % math symbols
\usepackage{amsthm} % thm environment
\usepackage{mdframed} % Customizable Boxes
\usepackage{hyperref,nameref,cleveref,enumitem} % for references, hyperlinks
\usepackage[dvipsnames]{xcolor} % Fancy Colours
\usepackage{mathrsfs} % Fancy font
\usepackage{tikz, tikz-cd, float} % Commutative Diagrams
\usepackage{perpage}
\usepackage{parskip} % So that paragraphs look nice
\usepackage{ifthen,xargs} % For defining better commands
\usepackage{anyfontsize}
\usepackage[T1]{fontenc}
\usepackage[utf8]{inputenc}
\usepackage{tgpagella}


% Shortcuts

% % Misc
\newcommand{\brkt}[1]{\left(#1\right)}
\newcommand{\sqbrkt}[1]{\left[#1\right]}
\newcommand{\dash}{\text{-}}

% % Logic
\renewcommand{\implies}{\Rightarrow}
\renewcommand{\iff}{\Leftrightarrow}
\newcommand{\limplies}{\Leftarrow}
\newcommand{\NOT}{\neg\,}
\newcommand{\AND}{\, \land \,}
\newcommand{\OR}{\, \lor \,}
\newcommand{\forwards}{$(\implies)$}
\newcommand{\backwards}{$(\limplies)$}

% % Sets
\DeclareMathOperator{\supp}{supp}
\newcommand{\set}[1]{\left\{#1\right\}}
\newcommand{\st}{\,|\,}
\newcommand{\minus}{\setminus}
\newcommand{\subs}{\subseteq}
\newcommand{\ssubs}{\subsetneq}
\DeclareMathOperator{\im}{Im}
\newcommand{\nothing}{\varnothing}

% % Greek 
\newcommand{\al}{\alpha}
\newcommand{\be}{\beta}
\newcommand{\ga}{\gamma}
\newcommand{\de}{\delta}
\newcommand{\ep}{\varepsilon}
\newcommand{\io}{\iota}
\newcommand{\ka}{\kappa}
\newcommand{\la}{\lambda}
\newcommand{\om}{\omega}
\newcommand{\si}{\sigma}

\newcommand{\De}{\Delta}
\newcommand{\Si}{\Sigma}

% % Mathbb
\newcommand{\N}{\mathbb{N}}
\newcommand{\Z}{\mathbb{Z}}
\newcommand{\Q}{\mathbb{Q}}
\newcommand{\R}{\mathbb{R}}
\newcommand{\C}{\mathbb{C}}
\newcommand{\F}{\mathbb{F}}
\newcommand{\bP}{\mathbb{P}}

% % Mathcal
\renewcommand{\AA}{\mathcal{A}}
\newcommand{\BB}{\mathcal{B}}
\newcommand{\CC}{\mathcal{C}}
\newcommand{\DD}{\mathcal{D}}
\newcommand{\EE}{\mathcal{E}}
\newcommand{\FF}{\mathcal{F}}
\newcommand{\GG}{\mathcal{G}}
\newcommand{\HH}{\mathcal{H}}
\newcommand{\II}{\mathcal{I}}
\newcommand{\JJ}{\mathcal{J}}
\newcommand{\KK}{\mathcal{K}}
\newcommand{\LL}{\mathcal{L}}
\newcommand{\MM}{\mathcal{M}}
\newcommand{\NN}{\mathcal{N}}
\newcommand{\OO}{\mathcal{O}}
\newcommand{\PP}{\mathcal{P}}
\newcommand{\QQ}{\mathcal{Q}}
\newcommand{\RR}{\mathcal{R}}
\renewcommand{\SS}{\mathcal{S}}
\newcommand{\TT}{\mathcal{T}}
\newcommand{\UU}{\mathcal{U}}
\newcommand{\VV}{\mathcal{V}}
\newcommand{\WW}{\mathcal{W}}
\newcommand{\XX}{\mathcal{X}}
\newcommand{\YY}{\mathcal{Y}}
\newcommand{\ZZ}{\mathcal{Z}}

% % Mathfrak
\newcommand{\f}[1]{\mathfrak{#1}}

% % Mathrsfs
\newcommand{\s}[1]{\mathscr{#1}}

% % Category Theory
\newcommand{\obj}[1]{\mathrm{Obj}\left(#1\right)}
\newcommand{\Hom}[3]{\mathrm{Hom}_{#3}(#1, #2)\,}
\newcommand{\mor}[3]{\mathrm{Mor}_{#3}(#1, #2)\,}
\newcommand{\End}[2]{\mathrm{End}_{#2}#1\,}
\newcommand{\aut}[2]{\mathrm{Aut}_{#2}#1\,}
\newcommand{\CAT}{\mathbf{Cat}}
\newcommand{\SET}{\mathbf{Set}}
\newcommand{\TOP}{\mathbf{Top}}
\newcommand{\GRP}{\mathbf{Grp}}
\newcommand{\AB}{\mathbf{Ab}}
\newcommand{\RING}{\mathbf{Ring}}
\newcommand{\MOD}[1][R]{#1\text{-}\mathbf{Mod}}
\newcommand{\VEC}[1][K]{#1\text{-}\mathbf{Vec}}
\newcommand{\ALG}[1][R]{#1\text{-}\mathbf{Alg}}
\newcommand{\PSH}[1]{\mathbf{PSh}\brkt{#1}}
\newcommand{\map}[4]{#1 \yrightarrow[#4][2.5pt]{#3}[-1pt] #2}
\newcommand{\op}{^{op}}
\newcommand{\darrow}{\downarrow}
\newcommand{\LIM}[2]{\varprojlim_{#2}#1}
\newcommand{\COLIM}[2]{\varinjlim_{#2}#1}

% % Algebra
\newcommand{\iso}{\cong}
\newcommand{\nsub}{\trianglelefteq}
\newcommand{\id}[1]{\mathrm{id}_{#1}}
\newcommand{\inv}{^{-1}}

% % Analysis
\newcommand{\abs}[1]{\left\vert #1 \right\vert}
\newcommand{\norm}[1]{\left\Vert #1 \right\Vert}
\renewcommand{\bar}[1]{\overline{#1}}
\newcommand{\<}{\langle}
\renewcommand{\>}{\rangle}
\renewcommand{\hat}[1]{\widehat{#1}}
\renewcommand{\check}[1]{\widecheck{#1}}

% % Galois
\newcommand{\Gal}[2]{\mathrm{Gal}_{#1}(#2)}
\DeclareMathOperator{\Orb}{Orb}
\DeclareMathOperator{\Stab}{Stab}
\newcommand{\emb}[3]{\mathrm{Emb}_{#1}(#2, #3)}
\newcommand{\Char}[1]{\mathrm{Char}#1}

% % Model Theory
\newcommand{\intp}[2]{
    \star_{\text{\scalebox{0.95}{$#1$}}}^{
    \text{\scalebox{0.7}{$#2$}}}}
\newcommand{\subintp}[3]{
    {#3}_{\text{\scalebox{0.95}{$#1$}}}^{
    \text{\scalebox{0.7}{$#2$}}}}
\newcommand{\modintp}[2]{#2^\text{\scalebox{0.7}{$#1$}}}
\newcommand{\mmintp}[1]{\modintp{\MM}{#1}}
\newcommand{\const}[1]{#1_\mathrm{con}}
\newcommand{\func}[1]{#1_\mathrm{fun}}
\newcommand{\rel}[1]{#1_\mathrm{rel}}
\newcommand{\term}[1]{#1_\mathrm{ter}}
\newcommand{\struc}[1]{#1_\mathrm{str}}
\newcommand{\form}[1]{#1_\mathrm{for}}
\newcommand{\var}[1]{#1_\mathrm{var}}
\newcommand{\theory}[1]{#1_\mathrm{the}}
\newcommand{\carrier}[1]{#1_\mathrm{car}}
\newcommand{\model}[1]{\vDash_{#1}}
\newcommand{\nodel}[1]{\nvDash_{#1}}

%% code from mathabx.sty and mathabx.dcl to get some symbols from mathabx
\DeclareFontFamily{U}{mathx}{\hyphenchar\font45}
\DeclareFontShape{U}{mathx}{m}{n}{
      <5> <6> <7> <8> <9> <10>
      <10.95> <12> <14.4> <17.28> <20.74> <24.88>
      mathx10
      }{}
\DeclareSymbolFont{mathx}{U}{mathx}{m}{n}
\DeclareFontSubstitution{U}{mathx}{m}{n}
\DeclareMathAccent{\widecheck}{0}{mathx}{"71}

% Arrows with text above and below with adjustable displacement
% (Stolen from Stackexchange)
\newcommandx{\yaHelper}[2][1=\empty]{
\ifthenelse{\equal{#1}{\empty}}
  % no offset
  { \ensuremath{ \scriptstyle{ #2 } } } 
  % with offset
  { \raisebox{ #1 }[0pt][0pt]{ \ensuremath{ \scriptstyle{ #2 } } } }  
}

\newcommandx{\yrightarrow}[4][1=\empty, 2=\empty, 4=\empty, usedefault=@]{
  \ifthenelse{\equal{#2}{\empty}}
  % there's no text below
  { \xrightarrow{ \protect{ \yaHelper[ #4 ]{ #3 } } } } 
  % there's text below
  {
    \xrightarrow[ \protect{ \yaHelper[ #2 ]{ #1 } } ]
    { \protect{ \yaHelper[ #4 ]{ #3 } } } 
  } 
}

% xcolor
\definecolor{darkgrey}{gray}{0.10}
\definecolor{lightgrey}{gray}{0.30}
\definecolor{slightgrey}{gray}{0.80}

% hyperref
\hypersetup{
      colorlinks = true,
      linkcolor = {blue},
      citecolor = {blue}
}

\newcommand{\link}[1]{\hypertarget{#1}{}}
\newcommand{\linkto}[2]{\hyperlink{#1}{#2}}

% Perpage
\MakePerPage{footnote}

% Theorems

% % custom theoremstyles
\newtheoremstyle{definitionstyle}
{5pt}% above thm
{0pt}% below thm
{}% body font
{}% space to indent
{\bf}% head font
{\vspace{1mm}}% punctuation between head and body
{\newline}% space after head
{\thmname{#1}\thmnote{\,\,--\,\,#3}}

\newtheoremstyle{exercisestyle}%
{5pt}% above thm
{0pt}% below thm
{\it}% body font
{}% space to indent
{\it}% head font
{.}% punctuation between head and body
{ }% space after head
{\thmname{#1}\thmnote{ (#3)}}

\newtheoremstyle{remarkstyle}%
{5pt}% above thm
{0pt}% below thm
{}% body font
{}% space to indent
{\it}% head font
{.}% punctuation between head and body
{ }% space after head
{\thmname{#1}\thmnote{\,\,--\,\,#3}}

% % Theorem environments

\theoremstyle{definitionstyle}
\newmdtheoremenv[
    linewidth = 2pt,
    leftmargin = 20pt,
    rightmargin = 20pt,
    linecolor = darkgrey,
    topline = false,
    bottomline = false,
    rightline = false,
    footnoteinside = true
]{dfn}{Definition}
\newmdtheoremenv[
    linewidth = 2 pt,
    leftmargin = 20pt,
    rightmargin = 20pt,
    linecolor = darkgrey,
    topline = false,
    bottomline = false,
    rightline = false,
    footnoteinside = true
]{prop}{Proposition}
\newmdtheoremenv[
    linewidth = 2 pt,
    leftmargin = 20pt,
    rightmargin = 20pt,
    linecolor = darkgrey,
    topline = false,
    bottomline = false,
    rightline = false,
    footnoteinside = true
]{cor}{Corollary}

\theoremstyle{exercisestyle}
\newmdtheoremenv[
    linewidth = 0.7 pt,
    leftmargin = 20pt,
    rightmargin = 20pt,
    linecolor = darkgrey,
    topline = false,
    bottomline = false,
    rightline = false,
    footnoteinside = true
]{ex}{Exercise}
\newmdtheoremenv[
    linewidth = 0.7 pt,
    leftmargin = 20pt,
    rightmargin = 20pt,
    linecolor = darkgrey,
    topline = false,
    bottomline = false,
    rightline = false,
    footnoteinside = true
]{eg}{Example}

\theoremstyle{remarkstyle}
\newtheorem{rmk}{Remark}

% tikzcd
% % Substituting symbols for arrows in tikz comm-diagrams.
\tikzset{
  symbol/.style={
    draw=none,
    every to/.append style={
      edge node={node [sloped, allow upside down, auto=false]{$#1$}}}
  }
}

\begin{document}
\title{preamble}
\author{author}
\date{Date}
\maketitle

\section{Name}

\begin{dfn}[Cardinalities of a signatures and structures]
    Given a signature $\Si$, 
    we define 
    $|\Si| := |\const{\Si}| + |\func{\Si}| + \rel{\Si}$
    to be the cardinality of the signature $\Si$.
\end{dfn}

\begin{prop}[Moving Models Down Signatures]
    Given two signatures such that $\const{\Si} \subs \const{\Si^*}$, 
    two theories $T \subs T^*$ such that $T$ is a $\Si$-theory and $T^*$ is a $\Si^*$-theory,
    given $\MM$ a $\Si^*$-model of $T^*$,
    then we can make $\MM$ into $(\carrier{\MM},\intp{\Si}{\MM})$
    a $\Si$-model of $T$.
    Technically the new structure is not $\MM$,
    but for convenience we write 
    $\MM$ to mean either of the two and let subscripts involving 
    $\Si$ and $\Si^*$ describe which one we mean.
\end{prop}
\begin{proof}
    Consider $(\carrier{\MM},\intp{\Si}{\MM})$, 
    where $\intp{\Si}{\MM}$ is defined by 
    $(\intp{\const{\Si}}{\MM},\intp{\func{\Si}}{\MM},\intp{\rel{\Si}}{\MM})$ 
    which are defined by:
    \begin{itemize}
        \item $\intp{\const{\Si}}{\MM}$ is
        the restriction of $\intp{\const{\Si^*}}{\MM}$ to $\const{\Si}$
        \item $\intp{\func{\Si}}{\MM}:=\intp{\func{\Si^*}}{\MM}$
        \item $\intp{\rel{\Si}}{\MM}:=\intp{\rel{\Si^*}}{\MM}$
    \end{itemize}
    We also want to have that for any $\Si$-term $t$, 
    the induced maps
    $\subintp{\Si}{\MM}{t}$ and $ \subintp{\Si^*}{\MM}{t}$
    are equal.
    Indeed:
    \begin{itemize}
        \item If $t$ is a constant then 
        $\subintp{\Si}{\MM}{t} = \subintp{\Si}{\MM}{c} =
        \subintp{\Si^*}{\MM}{c} = \subintp{\Si^*}{\MM}{t}$
        \item If $t$ is a variable then 
        $\subintp{\Si}{\MM}{t} = \id{\carrier{\MM}} =
        \subintp{\Si^*}{\MM}{t}$
        \item If $t$ is $f(s)$ then 
        $\subintp{\Si}{\MM}{t} = \subintp{\Si}{\MM}{f}(\subintp{\Si}{\MM}{s}) =
        \subintp{\Si^*}{\MM}{f}(\subintp{\Si^*}{\MM}{s}) = \subintp{\Si^*}{\MM}{t} $
    \end{itemize}
    
    Let $\phi$ be a formula in $T \subs T^*$ with variables
    indexed by $S \subs \N$.
    Let $a$ be in $(\carrier{\MM})^S$.
    Case on $\phi$ to show that 
    $\MM \model{\Si} \phi(a)$:
    \begin{itemize}
        \item If $\phi$ is $t = s$, 
        then $\MM \model{\Si^*} \phi(a)$.
        Hence by definition of $\intp{\Si}{\MM}$ we have
        $\subintp{\Si}{\MM}{t} = \subintp{\Si^*}{\MM}{t} =
        \subintp{\Si^*}{\MM}{s} = \subintp{\Si}{\MM}{s}$ 
        and so
        $\MM \model{\Si} \phi(a)$.
        \item If $\phi$ is $r(t)$, 
        then $\MM \model{\Si^*} \phi(a)$.
        Hence by definition of $\intp{\Si}{\MM}$ we have
        $\subintp{\Si}{\MM}{t}(a)  = \subintp{\Si^*}{\MM}{t}(a) \in
        \subintp{\Si^*}{\MM}{r} =  \subintp{\Si}{\MM}{r}$ 
        and so
        $\MM \model{\Si} \phi(a)$.
        \item If $\phi$ is $\NOT \psi$
        then $\MM \nodel{\Si^*} \psi(a)$.
        Hence $\MM \nodel{\Si} \psi(a)$ by the induction hypothesis.
        \item If $\phi$ is $\psi \OR \chi$
        then $\MM \model{\Si^*} \psi(a)$ or $\MM \model{\Si^*} \chi(a)$.
        Hence $\MM \model{\Si} \psi(a)$ or 
        $\MM \model{\Si} \chi(a)$
        by the induction hypothesis.
        \item If $\phi$ is $\forall v, \psi$ then for all $b \in \carrier{\MM}$, $\MM \model{\Si^*} \psi(a)$. 
        Hence for all $b \in \carrier{\MM}$,
        $\MM \model{\Si} \psi(a)$ by the induction hypothesis.
    \end{itemize}
    Hence $\MM \model{\Si} T$.
\end{proof}

\begin{prop}[Infinite models produce models with arbitrary large cardinality]
    Given $\Si$ a signature, 
    $T$ a $\Si$-theory that has an infinite model,
    and a cardinal $\ka \geq |\Si| + \aleph_0$, 
    there exists $\MM$ a $\Si$-model of $T$ such that 
    $\ka \leq |\carrier{\MM}|$.
\end{prop}
\begin{proof}
    Enrich only the signature's constant symbols to create $\Si^*$ 
    a signature such that 
    $\const{\Si^*} = \const{\Si} \cup \set{c_\al \st \al \in \ka}$.
    Let $T^* = T \cup \set{c_\al \neq c_\be \st \al,\be \in \ka \AND \al \ne \be}$
    be a $\Si^*$-theory.
    
    Using \linkto{compactness}{the compactness theorem}, 
    it suffices to show that $T^*$ is finitely satisfiable.
    Take a finite subset of $T^*$. 
    This is the union of a finite subset $\De_T \subs T$, 
    and a finite subset of 
    $\De_{\ka} \subs 
    \set{c_\al \neq c_\be \st \al,\be \in \ka \AND \al \ne \be}$.
    Let $\MM$ be the $\Si$-model of $T$ with infinite cardinality.
    We want to make 
    $(\carrier{\MM},\intp{\Si^*}{\MM}) \model{\Si^*} \De_T \cup \De_\ka$ 
    by interpreting the new symbols of $\set{c_\al \st \al \in \ka}$
    in a sensible way.
    
    Since $\De_\ka$ is finite, we can find a finite subset $I \subset \ka$ that indexes the constant symbols appearing in $\De_\ka$. 
    Since $\MM$ is infinite and $I$ is finite,
    we can find distinct elements of $\MM$
    to interpret the elements of
    $\set{c_\al \st \al \in I}$. 
    Interpret the rest of the new constant symbols however,
    for example let them all be sent to the same element,
    then $\MM \model{\Si*} \De_T \cup \De_\ka$.
    Hence $T*$ is satisfiable.
    
    Using 
    \linkto{better_compactness}{the even better compactness theorem},
    $T^*$ is finitely satisfiable implies there exists
    $\MM$ a $\Si^*$-model of $T^*$ with $|\carrier{\MM}|\leq \ka$.
    If $|\carrier{\MM}| < \ka$ 
    then there would be $c_\al, c_\be$ that are interpreted as equal,
    hence $\MM \model{\Si^*} c_\al = c_\be$ and $\MM \nodel{\Si^*} c_\al = c_\be$, 
    a contradiction.
    Thus $|\carrier{\MM}| = \ka$.
    \linkto{move_down_sig}{Move $\MM$ down a signature}
    to make it a model of $T$.
    This doesn't change the cardinality of $\MM$,
    so we have a model of $T$ with $|\carrier{\MM}| = \ka$.
\end{proof}
\end{document}