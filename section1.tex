\section{Name}

\begin{dfn}[Cardinalities of a signatures and structures]
    Given a signature $\Si$, 
    we define 
    $|\Si| := |\const{\Si}| + |\func{\Si}| + \rel{\Si}$
    to be the cardinality of the signature $\Si$.
\end{dfn}

\begin{prop}[Moving Models Down Signatures]
    Given two signatures such that $\const{\Si} \subs \const{\Si^*}$, 
    two theories $T \subs T^*$ such that $T$ is a $\Si$-theory and $T^*$ is a $\Si^*$-theory,
    given $\MM$ a $\Si^*$-model of $T^*$,
    then we can make $\MM$ into $(\carrier{\MM},\intp{\Si}{\MM})$
    a $\Si$-model of $T$.
    Technically the new structure is not $\MM$,
    but for convenience we write 
    $\MM$ to mean either of the two and let subscripts involving 
    $\Si$ and $\Si^*$ describe which one we mean.
\end{prop}
\begin{proof}
    Consider $(\carrier{\MM},\intp{\Si}{\MM})$, 
    where $\intp{\Si}{\MM}$ is defined by 
    $(\intp{\const{\Si}}{\MM},\intp{\func{\Si}}{\MM},\intp{\rel{\Si}}{\MM})$ 
    which are defined by:
    \begin{itemize}
        \item $\intp{\const{\Si}}{\MM}$ is
        the restriction of $\intp{\const{\Si^*}}{\MM}$ to $\const{\Si}$
        \item $\intp{\func{\Si}}{\MM}:=\intp{\func{\Si^*}}{\MM}$
        \item $\intp{\rel{\Si}}{\MM}:=\intp{\rel{\Si^*}}{\MM}$
    \end{itemize}
    We also want to have that for any $\Si$-term $t$, 
    the induced maps
    $\subintp{\Si}{\MM}{t}$ and $ \subintp{\Si^*}{\MM}{t}$
    are equal.
    Indeed:
    \begin{itemize}
        \item If $t$ is a constant then 
        $\subintp{\Si}{\MM}{t} = \subintp{\Si}{\MM}{c} =
        \subintp{\Si^*}{\MM}{c} = \subintp{\Si^*}{\MM}{t}$
        \item If $t$ is a variable then 
        $\subintp{\Si}{\MM}{t} = \id{\carrier{\MM}} =
        \subintp{\Si^*}{\MM}{t}$
        \item If $t$ is $f(s)$ then 
        $\subintp{\Si}{\MM}{t} = \subintp{\Si}{\MM}{f}(\subintp{\Si}{\MM}{s}) =
        \subintp{\Si^*}{\MM}{f}(\subintp{\Si^*}{\MM}{s}) = \subintp{\Si^*}{\MM}{t} $
    \end{itemize}
    
    Let $\phi$ be a formula in $T \subs T^*$ with variables
    indexed by $S \subs \N$.
    Let $a$ be in $(\carrier{\MM})^S$.
    Case on $\phi$ to show that 
    $\MM \model{\Si} \phi(a)$:
    \begin{itemize}
        \item If $\phi$ is $t = s$, 
        then $\MM \model{\Si^*} \phi(a)$.
        Hence by definition of $\intp{\Si}{\MM}$ we have
        $\subintp{\Si}{\MM}{t} = \subintp{\Si^*}{\MM}{t} =
        \subintp{\Si^*}{\MM}{s} = \subintp{\Si}{\MM}{s}$ 
        and so
        $\MM \model{\Si} \phi(a)$.
        \item If $\phi$ is $r(t)$, 
        then $\MM \model{\Si^*} \phi(a)$.
        Hence by definition of $\intp{\Si}{\MM}$ we have
        $\subintp{\Si}{\MM}{t}(a)  = \subintp{\Si^*}{\MM}{t}(a) \in
        \subintp{\Si^*}{\MM}{r} =  \subintp{\Si}{\MM}{r}$ 
        and so
        $\MM \model{\Si} \phi(a)$.
        \item If $\phi$ is $\NOT \psi$
        then $\MM \nodel{\Si^*} \psi(a)$.
        Hence $\MM \nodel{\Si} \psi(a)$ by the induction hypothesis.
        \item If $\phi$ is $\psi \OR \chi$
        then $\MM \model{\Si^*} \psi(a)$ or $\MM \model{\Si^*} \chi(a)$.
        Hence $\MM \model{\Si} \psi(a)$ or 
        $\MM \model{\Si} \chi(a)$
        by the induction hypothesis.
        \item If $\phi$ is $\forall v, \psi$ then for all $b \in \carrier{\MM}$, $\MM \model{\Si^*} \psi(a)$. 
        Hence for all $b \in \carrier{\MM}$,
        $\MM \model{\Si} \psi(a)$ by the induction hypothesis.
    \end{itemize}
    Hence $\MM \model{\Si} T$.
\end{proof}

\begin{prop}[Infinite models produce models with arbitrary large cardinality]
    Given $\Si$ a signature, 
    $T$ a $\Si$-theory that has an infinite model,
    and a cardinal $\ka \geq |\Si| + \aleph_0$, 
    there exists $\MM$ a $\Si$-model of $T$ such that 
    $\ka \leq |\carrier{\MM}|$.
\end{prop}
\begin{proof}
    Enrich only the signature's constant symbols to create $\Si^*$ 
    a signature such that 
    $\const{\Si^*} = \const{\Si} \cup \set{c_\al \st \al \in \ka}$.
    Let $T^* = T \cup \set{c_\al \neq c_\be \st \al,\be \in \ka \AND \al \ne \be}$
    be a $\Si^*$-theory.
    
    Using \linkto{compactness}{the compactness theorem}, 
    it suffices to show that $T^*$ is finitely satisfiable.
    Take a finite subset of $T^*$. 
    This is the union of a finite subset $\De_T \subs T$, 
    and a finite subset of 
    $\De_{\ka} \subs 
    \set{c_\al \neq c_\be \st \al,\be \in \ka \AND \al \ne \be}$.
    Let $\MM$ be the $\Si$-model of $T$ with infinite cardinality.
    We want to make 
    $(\carrier{\MM},\intp{\Si^*}{\MM}) \model{\Si^*} \De_T \cup \De_\ka$ 
    by interpreting the new symbols of $\set{c_\al \st \al \in \ka}$
    in a sensible way.
    
    Since $\De_\ka$ is finite, we can find a finite subset $I \subset \ka$ that indexes the constant symbols appearing in $\De_\ka$. 
    Since $\MM$ is infinite and $I$ is finite,
    we can find distinct elements of $\MM$
    to interpret the elements of
    $\set{c_\al \st \al \in I}$. 
    Interpret the rest of the new constant symbols however,
    for example let them all be sent to the same element,
    then $\MM \model{\Si*} \De_T \cup \De_\ka$.
    Hence $T*$ is satisfiable.
    
    Using 
    \linkto{better_compactness}{the even better compactness theorem},
    $T^*$ is finitely satisfiable implies there exists
    $\MM$ a $\Si^*$-model of $T^*$ with $|\carrier{\MM}|\leq \ka$.
    If $|\carrier{\MM}| < \ka$ 
    then there would be $c_\al, c_\be$ that are interpreted as equal,
    hence $\MM \model{\Si^*} c_\al = c_\be$ and $\MM \nodel{\Si^*} c_\al = c_\be$, 
    a contradiction.
    Thus $|\carrier{\MM}| = \ka$.
    \linkto{move_down_sig}{Move $\MM$ down a signature}
    to make it a model of $T$.
    This doesn't change the cardinality of $\MM$,
    so we have a model of $T$ with $|\carrier{\MM}| = \ka$.
\end{proof}